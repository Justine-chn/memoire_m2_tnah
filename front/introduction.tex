\chapter{Introduction} 

\par Les études sur les actes, les documents émanant d'une autorité et agissant en vertu de cette autorité\footnote{\cite{guyotjeanninDiplomatiqueMedievale2006}.}, ont dans un premier temps priorisé les actes pontificaux et royaux, \og laissant leurs homologues princiers au rang d'actes privés \fg  \footnote{Canteaut, Olivier, Moufflet, Jean-François, \og Les éditions d'actes princiers (\textsc{XII}\up{e} - \textsc{XV}\up{e} siècle) : bilan à l'ère du numérique \fg, in : \cite{guyotjeanninJeanBerryEcrit2019}.}. Les actes princiers désignent les actes émis par les princes. Le terme prince se doit d'être désambiguïsé dans ce travail. Bien que se rapportant d'ordinaire à l'héritier présomptif du trône royal au sens contemporain du terme, il ne doit ici pas être entendu pour nommer celui qui est premier par le sang ou par le rang, mais bien celui qui appartient à une maison souveraine et ne règne pas\footnote{Définition CNRTL :  \url{https://www.cnrtl.fr/definition/prince}.}. Les premiers projets consacrés aux actes princiers voient le jour au \textsc{XIX}\up{e} siècle, en Allemagne, offrant ainsi une analyse des pratiques entre princes, empereurs et évêques du Saint-Empire\footnote{\cite{bohmerRegestaImperiiInde1839}., \cite{hueberRegestenKaiserreichsUnter1877}.}. À l'instar de leurs voisins, les diplomatistes français se concentrent d'abord sur l'imitation des formes et des formules royales et c'est par ce biais que l'on voit émerger les prémices d'une diplomatique princière. Les premières recherches sont impulsées par le lancement, en 1894, par l'Académie des inscriptions et belles-lettres, d'une nouvelle série, en plus de celles dédiées aux actes royaux et pontificaux, consacrée aux actes des \og grands feudataires \fg\footnote{\cite{guyotjeanninJeanBerryEcrit2019}.}. Cette série s'ouvre avec la publication des actes d'Henri II Plantagenêt (1151-1189) relatifs à la France\footnote{\cite{delisleRecueilActesHenri1909}.}. Malheureusement, cette entreprise n'est pas poursuivie et il faut attendre la fin du XX\up{e} siècle pour assister au départ de nouveaux projets. La plupart de ces derniers se focalisent sur le début du bas Moyen Âge. On observe une raréfaction des éditions de documents des \textsc{XII}\up{e} - \textsc{XV}\up{e} siècle, à l'exception des cas bourguignon, breton et gascon\footnote{Annexe Répartition chronologique des éditions et catalogues d'actes princiers produits (espace français actuel), in Canteaut, Olivier, Moufflet, Jean-François, \og Les éditions d’actes princiers (\textsc{XII}\up{e} - \textsc{XV}\up{e} siècles) : bilan à l’ère du numérique \fg , in : \cite{guyotjeanninJeanBerryEcrit2019}.}. 

\par Force est de constater, en plus des disparités chronologiques, des disparités géographiques. Concernant les actes des ducs de Bretagne, nous pouvons citer les travaux de Michael Jones, pour les actes de Charles de Blois~(1341~-~1364), Jeanne de Penthièvre~(1341~-~1365, † 1384) et Jean \textsc{IV}~(1365~-~1399), de Marjolaine Lémeillat pour les actes de Pierre de Dreux~(1213~-~1221) et de Jean I\up{er}~(1221~-~1286), ainsi que ceux de Jean Kerhervé sur la chancellerie de Bretagne au temps du duc François \textsc{II}~(1458~-~1488), de sa fille la duchesse Anne~(1488~-~1514) et du premier mari de celle-ci, le roi de France Louis \textsc{XII}~(1498~-~1515) \footnote{\cite{jonesRecueilActesCharles2016}. \newline \cite{jonesRecueilActesJean1980}. \newline \cite{lemeillatActesPierreDreux2013}. \newline \cite{kerherveChancellerieBretagneSous2001}.}. Le duché de Bourgogne est également l'objet de travaux, parmi lesquels ceux d'Anne-Lise Courtel sur la Bourgogne pour les ducs capétiens, puis, d'Henri Stein pour les ducs Valois, Philippe le Hardi~(1363~-~1404), Jean sans Peur~(1404~-~1419) et Philippe le Bon~(1419~-~1467) et de Sonja Dünnebeil pour les actes de Charles le Téméraire~(1467~-~1477)\footnote{\cite{courtelChancellerieActesEudes1977} \newline \cite{steinCatalogueActesCharles1999}.}. Pierre Cockshaw a également étudié le personnel de la chancellerie de Bourgogne-Flandre\footnote{\cite{cockshawPersonnelChancellerieBourgogneFlandre1982}.}. Même si elles sont majoritaires, les entreprises d'édition ne concernent pas que les deux maisons précédemment citées. John Benton et Michel Bur ont étudié les actes du comte de Champagne, Henri le Libéral~(1152~-~1181)\footnote{\cite{bentonRecueilActesHenri2009}.}. La chancellerie dauphinoise sous Humbert~II~(1333~-~1349) a aussi engendré le dépouillement d'actes par Chantal Reydelet \footnote{\cite{reydeletChancellerieHumbertII1974}.}. Ces disparités géographiques et chronologiques s'expliquent par le fait que pour la période étudiée, les corpus sont importants et très dispersés. Nombre des travaux sur les actes princiers se sont limités à des analyses, à des catalogues, plutôt qu'à une édition complète face à l'abondance de la documentation. 
\newpage 

\par Toutefois, comme l'indique Olivier Canteaut, \og le développement de l'édition électronique scientifique et, plus globalement, les perspectives offertes par les humanités numériques stimulent en effet depuis plus d'une décennie la réalisation d'amples corpus d'actes et la diplomatique princière laïque bénéficie de ce regain d'intérêt \fg \footnote{Canteaut, Olivier, Moufflet, Jean-François, \og Les éditions d’actes princiers (\textsc{XII}\up{e} - \textsc{XV}\up{e} siècle) : bilan à l’ère du numérique \fg, in :\cite{guyotjeanninJeanBerryEcrit2019}.}. En effet, l'édition numérique offre l'opportunité d'entamer de vastes chantiers d'identification et de regroupement de sources et d'en publier un premier état. Cette opportunité n'est pas négligeable dans le cas où les données sont éclatées sur le territoire, dans différents services d’archives et nécessitent le travail conjoint de plusieurs chercheurs. Ce regain s'applique dans un premier temps à publier les analyses des actes\footnote{Les actes pontificaux en France, \url{https://qed.perspectivia.net/gallia-pontificia-online/}. \newline Regesta Imperii Online, \url{http://opac.regesta-imperii.de/lang_en/}.}, à d'autres supports comme les cartulaires et les chartriers \footnote{Digitale Monumenta Germaniae Historica \url{https://www.dmgh.de/}. \newline Cartulaires d'Île-de-France \url{http://elec.enc.sorbonne.fr/cartulaires/}. \newline Base des actes normands médiévaux Scripta \url{https://www.unicaen.fr/scripta/}. \newline Corpus de la Bourgogne au Moyen Âge CMBA \url{http://www.cbma-project.eu/}.}, mais aussi aux travaux déjà imprimés. Le projet \og Corpus des actes royaux de la France médiévale \fg \space (CARo) s'attache à rendre accessible sous forme numérique les actes des rois de France, de Charles le Chauve~(875~-~877) à Philippe Auguste~(1180~-~1223), édités par l’Institut de France\footnote{Corpus des actes royaux, en ligne : \url{https://lamop.hypotheses.org/category/corpus-des-actes-royaux-caro}.}. Ces entreprises sont motrices et participent au développement d'éditions dédiées aux actes des grands. C'est dans ce contexte que s'inscrit le projet consacré aux actes princiers. 
\newline 

\par Le projet \og Actes princiers au royaume de France (\textsc{XIV}\up{e} - \textsc{XVI}\up{e} siècle) \fg \space est porté par le professeur des universités Olivier Mattéoni et rassemble quatre institutions : deux unités mixtes de recherche (UMR) que sont le Laboratoire de Médiévistique occidentale de Paris (LaMOP, Paris 1 - CNRS\footnote{Laboratoire de médiévistique occidentale de Paris, \url{https://lamop.pantheonsorbonne.fr/}.}), qui porte le projet et le Centre de recherches historiques (CRH, CNRS - EHESS), le centre de recherche de l'École Nationale des Chartes, Centre Jean Mabillon, et le département du Moyen Âge et de l’Ancien régime des Archives nationales. L'ambition du projet est de procéder à l’édition numérique de plusieurs corpus d’actes inédits issus des chancelleries des princes de sang de la fin du Moyen Âge. Ces corpus comprennent les actes des ducs et duchesses de Bourbon et du duc Jean de Berry~(1360~-~1416). Leur nombre est voué à s'accroître et à comporter par la suite d'autres actes princiers comme ceux de la maison d'Orléans et de la maison d'Anjou. Le projet comprend le développement d’un site web et l’encodage d’une première série d’actes : ceux de Louis \textsc{II}~(1356~-~1410) et de Charles I\up{er}~(1434~-~1456) de Bourbon et de leurs épouses Anne Dauphine~(1371~-~1410, † 1417) et Agnès de Bourgogne~(1434~-~1456, † 1476)\footnote{Annexe Généalogie de la famille ducale de Bourbon.}. 
\newpage 

\par Louis II de Bourbon devient duc en 1356, dans des conditions difficiles. À la suite de la défaite de Poitiers, il est envoyé comme otage en Angleterre en échange de la libération du roi de France Jean II~(1350~-~1364)\footnote{\og Poitiers\fg, in : \cite{favierGuerreCentAns1980}, p.181.}. Ses années de captivité entravent considérablement la gouvernance du duché. À son retour, en 1366, il soude la noblesse autour de l’ordre de chevalerie de l’écu d’or et mène une politique d'expansion territoriale par le biais d’alliances. Son mariage avec Anne de Forez, héritière du comté de Forez, en 1371, sanctionne le rattachement du comté à la principauté bourbonnaise. En 1400, le mariage de son fils Jean de Bourbon~(1410~-~1434) avec Marie de Berry~(1416~-~1434) prévoit l’intégration du duché d’Auvergne à la mort du duc Jean de Berry, intégration qui sera effective en 1425. L’Auvergne est un apanage qui doit revenir au domaine royal dans le cas où le duc, père de Marie de Berry, n’a pas d’héritier mâle. Toutefois, Jean de Berry et Louis II de Bourbon obtiennent du conseil royal et de Charles VI~(1380~-~1422) que cet apanage revienne aux Bourbons du fait du mariage de leurs enfants. Cette expansion participe au développement institutionnel notable du duché impulsé par le duc, qui se traduit par d’importantes réformes avec la création d’un office de trésorier général (1372) et d'une chambre des comptes (1374)\footnote{\og Les ducs de Bourbon, le roi et le royaume de France à la fin du Moyen Âge \fg, in : \cite{matteoniBourbonsLeurBibliotheque2022}.}. Olivier Mattéoni parle d'une \og fièvre normative \fg \space pour les domaines des finances et de la justice\footnote{\cite{matteoniProposOrdonnanceFinance2019}.}. L'agrandissement de la principauté et les réformes qui en découlent génèrent une importante production documentaire d'actes écrits servant à créer ou confirmer des actions juridiques\footnote{\cite{guyotjeanninDiplomatiqueMedievale2006}.}. La production de ces actes s’inscrit dans un projet de réforme du gouvernement rendu nécessaire par l'agrandissement de la principauté. Les années 1370 sont marquées par une réorganisation de la chancellerie avec l’augmentation du nombre de secrétaires et une rationalisation des règles de confection des actes avec notamment la généralisation et l’obligation de la signature\footnote{\cite{matteoniEcriturePouvoirPrincier2011}.}. 
\newline 

\par En 1410, Jean I\textsuperscript{er} succède à son père à la tête d’une principauté bien administrée. Il est duc de 1410 à 1434, mais exerce peu la réalité du pouvoir, car il est fait prisonnier par les Anglais en 1415 à Azincourt\footnote{\og Les ducs de Bourbon, le roi et le royaume de France à la fin du Moyen Âge \fg, in : \cite{matteoniBourbonsLeurBibliotheque2022}.}. La production documentaire du duc Jean I\up{er} et de son épouse Marie de Berry est exclue de cette étude. Le travail de recensement des actes s'est concentré sur ceux de Louis II et Charles I\up{er} et de leurs épouses du fait de leurs réformes dans l'administration bourbonnaise et des acquisitions territoriales importantes sous leurs règnes. De plus, les actes dénombrés pour le couple Jean I\up{er} et Marie de Berry sont très peu nombreux : seuls cinquante-neuf actes ont pu être recensés (et quarante-quatre d'entre eux sont établis avant 1415)\footnote{\cite{generoChancellerieCharlesIer2018}.}. Pendant la captivité du duc, les femmes administrent le duché : d’abord Anne Dauphine, comme duchesse douairière jusqu’en 1417, puis Marie de Berry. La période est caractérisée par la réconciliation avec la maison de Bourgogne qui se concrétise par l'alliance matrimoniale entre Charles de Bourbon et Agnès de Bourgogne, la fille du duc de Bourgogne Jean Sans Peur, en 1425. 
\newline 

\par Charles I\up{er} devient duc en 1434, mais en raison de la captivité de son père, il assure le gouvernement du duché avant. Il bénéficie d'une proximité importante avec le roi de France Charles VII~(1422~-~1461) qui le nomme lieutenant en Languedoc et dans les marches est du royaume et dont il est le conseiller dès les années 1420\footnote{\cite{leguaiSeigneurieEtatBourbonnais1975a}.}. Il est également choisi par ce dernier pour diriger la délégation royale à la paix d’Arras, en 1435. Son mariage avec Agnès de Bourgogne le met en possession du duché d'Auvergne, ce qui contribue à affermir son influence. Il noue des alliances avec les autres princes du royaume et entre en rébellion contre le roi lors de la Praguerie en 1440\footnote{En 1440, révolte des grands contre Charles VII (à laquelle se joignit le Dauphin).}. Après l'échec de cette dernière, il se concentre sur l'administration et les finances avec la création de l'office de contrôleur général. Son principat est caractérisé par le renforcement des cadres institutionnels de la principauté (finances, fiscalité… )\footnote{\og Les ducs de Bourbon, le roi et le royaume de France à la fin du Moyen Âge\fg, in : \cite{matteoniBourbonsLeurBibliotheque2022}.}. Il cherche également à s'assurer le soutien de la noblesse, comme le montre la création de l'Armorial de Bourbonnais et de Forez\footnote{Armorial : Registre ou catalogue contenant les armes ou armoiries de la noblesse d'un royaume, d'une province, d'une ville, d'une famille, dessinées, peintes ou décrites. Définition CNRTL. \cite{deboosArmorialAuvergne1998}.}. La cour des ducs de Bourbon, implantée à Moulin, devient alors un centre artistique accueillant sculpteurs, musiciens, écrivains\footnote{\cite{leguaiSeigneurieEtatBourbonnais1975a}.}. L'épouse de Charles I\up{er}, Agnès de Bourgogne, n'est pas sans jouer un rôle diplomatique. Elle fait régulièrement l'intermédiaire entre son mari et son frère, et s'attache à mener une politique de pacification entre les deux familles \footnote{\cite{leguaiAgnesBourgogneDuchesse1996}.}. Sept de ses onze enfants sont notamment élevés à la cour de Bourgogne. La fin du principat de Charles I\up{er} est marquée par sa maladie. Il meurt le 4 décembre 1456 et son épouse lui survit jusqu'en 1476.
\newpage 

\par L'objectif du stage dans le projet « Actes princers » est de participer à la conception d'une chaîne de traitement afin de transformer des fichiers ODT contenant les éditions diplomatiques d’actes bourbonnais en des fichiers XML (TEI), pour assurer leur mise en ligne. L'un des enjeux est de tester les différentes solutions de transformation pour les documents afin de déterminer la plus adaptée au corpus, et pour des besoins futurs. Ensuite, les métadonnées associées aux corpus sont extraites via des scripts Python afin de les incorporer dans les fichiers XML/TEI produits. Ces derniers sont intégrés à une application Flask prototypée pour leur mise en ligne. 
\newline 

\par Ce travail est accompagné d'un livrable technique situé dans un repository GitLab\footnote{Justine Chainiau, Chaîne de traitement des actes princiers, en ligne : \url{https://gitlab.huma-num.fr/medieval-acts/princely-acts/stage_actes_princiers.git}.}. Il comprend les fichiers contenant les actes édités sous différents formats (ODT, PDF, XML) ainsi que les fichiers CSV contenant les métadonnées des actes. Il comprend également une chaîne de traitement de données (scripts Python) codée par les ingénieurs du projet, fonctionnelle et documentée. Il est complété par le descriptif des étapes du travail et du code nécessaire à la réflexion sur l’édition numérique et la mise en ligne d'actes princiers. 
\newline 

\par L'enjeu de ce travail de recherche autour du projet \og Actes Princiers \fg \space est de contribuer à une meilleure connaissance de la diplomatique princière qui est encore obscure pour la fin du Moyen Âge en France. 
\par Ce projet d'édition s'articule d'abord autour de l'édition des actes dispersés des princes de Bourbon, Louis II, Anne Dauphine, Charles I\up{er} et Agnès de Bourgogne. 
\par Il s'accompagne ensuite d'une importante réflexion sur la transformation des éditions, du traitement de texte vers l'encodage, afin de choisir les opérations de préparation, le langage d'encodage et le convertisseur les plus adaptés. 
\par Enfin, l'aboutissement du projet réside dans la conception d'une chaîne de traitement des actes via la technologie Python, afin de les pourvoir d'une structuration optimale nécessaire à la mise en ligne des données. 

\newpage
\thispagestyle{empty}
\mbox{}
\newpage