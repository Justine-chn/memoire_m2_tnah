\chapter{Conclusion}

\par La mobilisation de techniques numériques a permis de faciliter l'étude des corpus conséquents et dispersés que sont ceux de Louis II, d'Anne Dauphine, de Charles I\up{er} et d'Agnès de Bourgogne. 
\newline 

\par Le présent mémoire s'est attaché à présenter les documents étudiés : datation, typologie, dispersion, similitudes des actes des ducs de Bourbon. Nous sommes également revenus sur la manière dont ces derniers ont été traités initialement, car le projet d'encodage des actes pour la mise en ligne ne part pas de rien. En effet, les actes ont d'abord été édités avec un logiciel de traitement de texte. Nous avons alors développé une chaîne de traitement des documents à partir des éditions utilisées comme \textit{legacy data}. Ensuite, nous avons travaillé sur les documents afin de les préparer à l'encodage en XML/TEI. Cette étape s'est articulée autour d'une proposition de méthode de stylage des éditions en vue d'une conversion, puis du choix d'un convertisseur approprié. Puis, après conversion des documents, plusieurs scripts Python ont facilité les interventions sur les éditions de façon à améliorer et affiner leur structuration. Ces étapes ont également permis d'extraire et d'ajouter des données. Enfin, les fichiers ont été préparés à la mise en ligne via la génération d'un fichier XML pour chaque acte, puis une compilation en PDF (via une transformation en LaTeX). Ces dernières étapes ont rendu visibles les anomalies et possibles les corrections sur le document et initié la réflexion sur le projet (intégration d'actes inédits ...). 
\newline 

\par Toutes les opérations de traitement des actes explicitées dans ce travail peuvent être qualifiées de semi-automatiques, dans la mesure où il y a une alternance entre des traitements automatiques et des traitements manuels. Sébastien de Valeriola insiste sur la complémentarité de ces deux méthodes en affirmant que \og le travail de la machine ne remplace aucunement celui de l’historien, mais l’accompagne et le facilite \fg \footnote{\cite{devaleriolaOrdinateurAuService2020}.}. L'enjeu de ce travail sur le projet \og Actes Princiers \fg \space était d'automatiser ce qui pouvait l'être. Comme nous partions d'éditions réalisées avec un logiciel de traitement de texte, de nombreuses opérations ont été semi-automatiques, comme le stylage des actes via des regex en première intention et manuel pour les cas les plus compliqués. L'ordinateur est un bon outil pour réaliser les traitements les plus volumineux comme la transformation en XML et la séparation des corpus de chaque prince en fichiers individuels. Cette automatisation évite les erreurs humaines d'encodage dans notre cas. Mais, il est nécessaire de contrôler l'issue de ces manipulations, car il est fréquent que la machine bute sur des erreurs antérieures. L'automatisation permet également de rendre applicable le processus de traitement des données à l'avenir à d'autres actes et éventuels corpus. Toutefois, malgré sa réplicabilité, l'ensemble d'opérations développé est adapté à un corpus relativement homogène. Il ne faut pas perdre de vue qu'il est difficile de proposer une méthodologie cohérente, car même en suivant un modèle d'édition précis et reconnu, on ne peut pas prévenir toutes les individualités inhérentes à la recherche. L'enjeu reste donc de tendre au plus uniforme possible.
\newline 

\par L'uniformité, lorsqu'elle est atteinte, ou du moins approchée au maximum, entraîne une automatisation accrue des procédés de traitement des données et facilite ainsi l'extraction d'un maximum d'informations des corpus. Dans ce sens, la méthode de stylage des documents pourrait être encore affinée et couplée à de la diplomatique en distinguant les différentes parties de l'acte (intitulation ...), afin d'étudier leurs caractéristiques via des méthodes quantitatives et ainsi de comparer les formules utilisées, la rédaction et l'utilisation de modèles dans certains cas ou encore l'évolution des pratiques d'écriture. Des études prosopographies pourraient également être développées à partir des personnes référencées dans chaque corpus et ainsi mettre en lumière des réseaux sociaux. En effet, les actes sont riches en renseignement sur les acteurs dans la mesure où ils mentionnent souvent une activité, un lieu d'habitation, parfois des relations. Plus largement, le site pourrait être enrichi d'autres corpus. Dans un premier temps, ceux de Jean I\up{er} et de Marie de Berry pourraient bénéficier d'un travail dans la mesure où ils n'ont pas encore été très approfondis et où il serait pertinent de faire le lien entre les deux périodes de ducat de Louis II et de Charles I\up{er}. Cela permettrait aussi d'étudier l'administration du duché pendant la période particulière que fut celle de la captivité du duc, en pleine guerre de Cent Ans. 
\newline 

\par Le projet \og Actes Princiers\fg \space et les travaux menés dessus donnent l'impulsion aux recherches sur les actes des princes du royaume de France. Enfin, ce type de projet rend les données disponibles sur le web et accessibles à la communauté scientifique en proposant des \textit{legacy data} homogènes et facilement exploitables et contribue ainsi à une meilleure connaissance de la diplomatique princière pour la période étudiée.

\newpage
\thispagestyle{empty}
\mbox{}
\newpage